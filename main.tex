%!TEX program = xelatex
\documentclass[10pt]{beamer}
\usetheme{Darmstadt}
\usecolortheme{orchid}
\useinnertheme{rectangles}
\usepackage{eso-pic}
\usepackage{graphicx}
\usepackage{hyperref}
\usepackage{fontawesome5}
\usepackage{tabularx}
\usepackage{booktabs}
\usepackage{NWPU}


\begin{document}

% 封面页(带联系方式)
\begin{frame}[plain]
  \titlepage
\end{frame}

% 目录页
\begin{frame}[plain]{Outlines}
  % 目录页自定义样式
  \setbeamercolor{background canvas}{bg=lightpurple}
  \tableofcontents
\end{frame}

% ...existing code...
\section{Linear Time-Invariant System (LTI) Test}
\begin{frame}{Linear Time-Invariant System (LTI) Overview}
  \begin{block}{Definition}
    A Linear Time-Invariant (LTI) system is a system whose output response to an input is linear and does not change over time.
  \end{block}
  \begin{block}{Mathematical Representation}
    The output $y(t)$ of an LTI system is given by the convolution:
    \[
      y(t) = (x * h)(t) = \int_{-\infty}^{\infty} x(\tau) h(t-\tau) d\tau
    \]
    where $x(t)$ is the input and $h(t)$ is the impulse response.
  \end{block}
\end{frame}

\begin{frame}[fragile]{Python Example: LTI System Simulation}
  \begin{block}{Code Example}
    \lstset{
      language=Python,
    }
    \begin{lstlisting}
import numpy as np
import matplotlib.pyplot as plt
from scipy.signal import lti, step

system = lti([1], [1, 1])  # H(s) = 1/(s+1)
t, y = step(system)
plt.plot(t, y)
plt.xlabel('Time (s)')
plt.ylabel('Response')
plt.title('Step Response of LTI System')
plt.grid(True)
plt.show()
    \end{lstlisting}
  \end{block}
\end{frame}

\begin{frame}{LTI System Illustration}
  \centering
  \includegraphics[width=0.7\textwidth]{logo/logo.png}
  \captionof{figure}{校徽图示}
\end{frame}

\begin{frame}{LTI System Properties Table}
  \begin{block}{Comparison of System Properties}
    \begin{tabularx}{\textwidth}{lcc}
      \toprule
      \textbf{Property} & \textbf{LTI System} & \textbf{Non-LTI System} \\
      \midrule
      Linearity & Yes & Not always \\
      Time-Invariance & Yes & Not always \\
      Superposition & Holds & Not always \\
      Impulse Response & Exists & Not always meaningful \\
      Frequency Response & Well-defined & Not always \\
      \bottomrule
    \end{tabularx}
  \end{block}
\end{frame}
% ...existing code...


% 结语(带联系方式)
\section{The End}
\begin{frame}{Thank You!}
  \centering
  \Huge{\bfseries Any Questions?} 
  \vskip0.3cm
    \begin{tikzpicture}[inner sep=0,outer sep=0]
      \draw [black,line width=1.2pt,scope fading=fadetitleliner] (0,0) -- (\textwidth,0.0cm);
    \end{tikzpicture}%
    \vskip0.3cm
    \small
  \faEnvelope\ \url{zhangsan@gmail.com} \quad

  \faGithub\ \url{github.com/zhangsan} \quad

  \faPhone\ \url{(+86) 123-456-7890} \\[1cm]
  \vfill
 \begin{tikzpicture}[remember picture,overlay]
    \node[anchor=south, xshift=0cm, yshift=0cm, opacity=0.25] at (current page.south) {
      \includegraphics[width=\paperwidth]{logo/footbg.png} % 替换为你的图片路径
    };
  \end{tikzpicture}
\end{frame}

\end{document}